% !TEX root = ../thesis.tex


\abstract{%
  Digital micromirror devices (DMDs) are effective tools for the creation of arbitrarily shaped optical potentials. In a recent experiment in our group it was discovered that spurious reflections decrease the quality of the projected pattern. In the first part of this thesis I describe an algorithm that could reduce this kind of deviation. A CCD camera is used to provide intensity based feedback to the DMD. To do this, a robust calibration method for mapping the camera coordinates onto the DMD is developed first. Using this, we can correct for small scale aberrations. With randomised target patterns, we can regularly achieve final RMS errors of $<\SI{1}{\percent}$.

  The study of ultracold gases in uniform potentials has traditionally been achieved by evaporatively cooling the sample in a harmonic potential and then transferring it to the desired uniform potential. This method is however accompanied by transfer losses and heating. In the second part, I present a three-dimensional numerical simulation of gas dynamics during evaporative cooling which shows that efficient evaporative cooling is possible in a uniform potential if the atomic cloud is compressed during the evaporation sequence. Our simulated gases can retain $\num{2e6}$ atoms at a phase space density of 1.
}
