% !TEX root = ../thesis.tex


\chapter{Introduction}
% \label{cha:introduction}

Since the production of the first Bose-Einstein condensate (BEC) in the groups of Wieman and Cornell, and shortly after in the group of Ketterle in 1995 \cite{bec1,bec2}, there have been tremendous developments in the field of ultracold atoms. Today, many different elements have been condensed \cite{PhysRevLett.78.985,PhysRevLett.81.3811,PhysRevLett.85.1795,Modugno1320,PhysRevLett.86.3459,PhysRevLett.91.040404,Weber232,PhysRevLett.94.160401,PhysRevLett.103.130401,PhysRevA.82.011608,PhysRevLett.108.210401,PhysRevLett.107.190401} and with more advanced techniques, large BECs can now be produced within a short amount of time \cite{Rudolph2015,PhysRevA.95.013609}. Even photons have been condensed into a BEC~\cite{photonbec}.

With better understanding and new experiments, more and more questions can be answered, but so arise new questions. Experimental highlights are for example the BEC-BCS crossover \cite{PhysRevLett.92.120403,PhysRevLett.92.040403}, the superfluid to Mott-insulator phase transition \cite{mottInsulator}, the BKT crossover \cite{bktCrossover}, non-equilibrium physics \cite{nonEquilibrium} and effects of lower dimensionality \cite{oneDim1,oneDim2}.

Such a wide variety of experiments is possible due to precise control of the Hamiltonian. Specifically, the interaction and potential terms can be strongly influenced with the use of Feshbach resonances \cite{RevModPhys.82.1225} and the shaping of trapping potentials into various forms and with various strengths.
This thesis is going to focus on the latter of these two parts, the creation of potentials for the purpose of atom trapping, specifically optical box potentials. Among all possible potential geometries, the uniform or box potential holds a special place as it provides the researcher with the ability to study atoms without the influence of external electromagnetic fields.\\[\baselineskip]
The first part of this thesis focuses on the correction of disorder that disturbs the otherwise flat potential in one of our group's experiments. The use of digital micromirror devices allows us to achieve microscopic control over the intensity of a light field in a plane.

In the second part, we move away from an already existing experiment and instead explore the overall shaping and control of uniform optical potentials for a new experiment. To this end, we study the applicability of uniform potentials for evaporative cooling.


% This thesis deals with two different approaches to the creation of uniform box potentials. In the first part we propose a possible solution to a problem one of our experiments was met with when using digital micromirror devices for the production of Bose gases in a two-dimensional optical box.

% The second part explores evaporative cooling inside of a three-dimensional optical box potential. While evaporation in non-uniform potentials relies on the increase in density to achieve increasing (or at least constant) scattering rates, this is not possible for direct evaporation in a stationary box potential. We can show that by dynamically altering the size of the box, high phase space densities and high atom numbers can be achieved on relatively short timescales.